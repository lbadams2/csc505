\documentclass{article}
\usepackage[utf8]{inputenc}

\title{CSC505}
\author{Liam Adams}
\date{January 27 2019}

\begin{document}

\maketitle

\section{Unique1 Analysis}
The algorithm is implemented in unique1.cpp in the function \textit{getLongestSubsequence}. Prior to calling that function, a function called \textit{readInput} scans the file and processes it into a vector.  Processing the file takes $O(n)$ time.\\
There are 3 nested for loops in \textit{getLongestSubsequence} - inside each of which comparisons, assignments, and arithmetic operations are performed in constant time.  In the worst case scenario, each loop can run n times.  Therefore the worst case running time is $O(n^3)$.\\
The $n^3$ term dominates the runtime as n approaches infinity, so the overall worst case running time of the program is $O(n^3)$\\

\section{Unique2 Analysis}
The algorithm is implemented in unique2.cpp in the function \textit{getLongestSubsequence}. Prior to calling that function, a function called \textit{readInput} scans the file and processes it into a vector.  Processing the file takes $O(n)$ time.\\
There are 2 nested for loops in \textit{getLongestSubsequence} - inside each of which comparisons, assignments, and arithmetic operations are performed in constant time.  Inside the second loop a search on a balanced binary tree is performed, which takes $O(lg(n))$ time.  In the worst case scenario, each loop can run n times.  Therefore the worst case running time is $O(n^2lg(n))$.\\
The $n^2$ term dominates the runtime as n approaches infinity, so the overall worst case running time of the program is $O(n^2lg(n))$
\end{document}
