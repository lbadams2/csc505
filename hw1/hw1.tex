\documentclass{article}
\usepackage[utf8]{inputenc}
\usepackage{csc505}
\usepackage{myhandout}

\title{CSC505 HW1}
\author{Liam Adams}
\date{January 27 2019}

\begin{document}

\maketitle

\section*{Problem 1}
The sorted order where $g_i\epsilon o(g_{i+1})$ or $g_i \epsilon\Theta(g_{i+1})$ is $$n^{.0001}, lg^2n, n^2, n^{lgn}, 2^n, (n-1)!, n!, n^n$$ I will prove each relationship from left to right. The transitivity of little-o and big-theta allow me to only prove the adjacent relationships.\\\\
$\lim_{n \to \infty}\frac{n^\frac{1}{10000}}{lg^2n} = \frac{1}{n^{10000}lg^2n} = 0$\\
$n^{.0001}=o(lg^2n)$\\\\
$\lim_{n \to \infty}\frac{lg^2n}{n^2}=\frac{\frac{d}{dn}(lgn\times lgn)}{\frac{d}{dn}n^2}=\frac{2lgn}{n^3ln2}=0$\\
$lg^2n=o(n^2)$\\\\
For any positive constant $c > 0$, there exists a constant $n_0>0$ such that $0 \leq n^2 < cn^{lgn}$ for all $n \geq n_0$. For $n_0>4$, $n^2 < cn^{lgn}$ with c = 1. For $n_0<4$, a sufficiently large c can satisfy the preceding inequality. Therefore $n^2=o(n^{lgn})$\\\\
For any positive constant $c > 0$ there exists a constant $n_0>0$ such that $0 \leq n^{lgn} < c2^n$ for all $n \geq n_0$. $\lim_{n \to \infty}\frac{lgn}{n}=\frac{\frac{d}{dn}(lgn)}{\frac{d}{dn}n}=\frac{1}{nln2}=0$. Because $lgn=o(n)$, $n^{lgn}=o(2^n)$\\\\
For any positive constant $c > 0$ there exists a constant $n_0>0$ such that $0 \leq 2^{n} < c(n-1)!$ for all $n \geq n_0$. For $n_0>4$, $2^n < c(n-1)!$ with c = 1. For $n_0<4$, a sufficiently large c can satisfy the preceding inequality. Therefore $2^{n}=o((n-1)!)$\\\\
$\lim_{n \to \infty}\frac{(n-1)!}{n!}=\frac{1}{n}=0$\\
$(n-1)!=o(n!)$\\\\
For any positive constant $c > 0$ there exists a constant $n_0>0$ such that $0 \leq n! < cn^n$ for all $n \geq n_0$. For $n_0>3$, $n! < cn^n$ with c = 1. For $n_0<3$, a sufficiently large c can satisfy the preceding inequality. Therefore $n!=o(n^n)$\\\\
\section*{Problem 4}
4a) the for loop in unusual puts elements in different halves of the array, otherwise elements in different halves would never be compared.
Correctness: 
    Initialization: Cruel does not call Unusual until n=2, when both halves of A are trivially sorted. 
    
    Maintenance: Unusual then gradually sorts A as the stack of Cruel unwinds and n grows. The for loop swaps the second quarter of the array with the 
    third quarter of the array.  The recursive calls to Unusual maintain the invariant that each half is sorted, as the quarters are now 
    halves for the recursive calls.
    
    Termination: As the stack of Cruel unwinds, Unusual gradually sorts A in n_i=2^i chunks, until i=k where n=2^k.  At this point the entire array
    has been sorted.

4b) The array [3 4 1 2] would not sort correctly if the for loop were removed.  Unusual would return [3 1 4 2]

4c) The array [3 4 1 2] would not sort correctly if the last 2 lines were swapped.  Unusual would return [1 3 2 4]
\end{document}