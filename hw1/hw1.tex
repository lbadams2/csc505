\documentclass{article}
\usepackage[utf8]{inputenc}
\usepackage{csc505}
\usepackage{myhandout}
\usepackage{mathtools}
\DeclarePairedDelimiter\ceil{\lceil}{\rceil}
\DeclarePairedDelimiter\floor{\lfloor}{\rfloor}
\makeatletter% Set distance from top of page to first float
\setlength{\@fptop}{5pt}
\makeatother

\title{CSC505 HW1}
\author{Liam Adams}
\date{January 27 2019}

\begin{document}

\maketitle

\section*{Problem 1}
The sorted order where $g_i\epsilon o(g_{i+1})$ or $g_i \epsilon\Theta(g_{i+1})$ is $$n^{.0001}, lg^2n, n^2, n^{lgn}, 2^n, (n-1)!, n!, n^n$$ I will prove each relationship from left to right. The transitivity of little-o and big-theta allow me to only prove the adjacent relationships.\\\\
$\lim_{n \to \infty}\frac{n^\frac{1}{10000}}{lg^2n} = \frac{1}{n^{10000}lg^2n} = 0$\\
$n^{.0001}=o(lg^2n)$\\\\
$\lim_{n \to \infty}\frac{lg^2n}{n^2}=\frac{\frac{d}{dn}(lgn\times lgn)}{\frac{d}{dn}n^2}=\frac{2lgn}{n^3ln2}=0$\\
$lg^2n=o(n^2)$\\\\
For any positive constant $c > 0$, there exists a constant $n_0>0$ such that $0 \leq n^2 < cn^{lgn}$ for all $n \geq n_0$. For $n_0>4$, $n^2 < cn^{lgn}$ with c = 1. For $n_0<4$, a sufficiently large c can satisfy the preceding inequality. Therefore $n^2=o(n^{lgn})$\\\\
For any positive constant $c > 0$ there exists a constant $n_0>0$ such that $0 \leq n^{lgn} < c2^n$ for all $n \geq n_0$. $\lim_{n \to \infty}\frac{lgn}{n}=\frac{\frac{d}{dn}(lgn)}{\frac{d}{dn}n}=\frac{1}{nln2}=0$. Because $lgn=o(n)$, $n^{lgn}=o(2^n)$\\\\
For any positive constant $c > 0$ there exists a constant $n_0>0$ such that $0 \leq 2^{n} < c(n-1)!$ for all $n \geq n_0$. For $n_0>4$, $2^n < c(n-1)!$ with c = 1. For $n_0<4$, a sufficiently large c can satisfy the preceding inequality. Therefore $2^{n}=o((n-1)!)$\\\\
$\lim_{n \to \infty}\frac{(n-1)!}{n!}=\frac{1}{n}=0$\\
$(n-1)!=o(n!)$\\\\
For any positive constant $c > 0$ there exists a constant $n_0>0$ such that $0 \leq n! < cn^n$ for all $n \geq n_0$. For $n_0>3$, $n! < cn^n$ with c = 1. For $n_0<3$, a sufficiently large c can satisfy the preceding inequality. Therefore $n!=o(n^n)$\\\\
\section*{Problem 3}
a) Case 1 of the Master Theorem applies here.  $f(n)=n^2lgn$ and $n^{\log_ba}=n^{4.03}$.  $f(n)=O(n^{4.03-\epsilon})$ for $\epsilon=1$, therefore $T(n)=\Theta(n^{\log_4 17})$\\\\
b) The Master Theorem does not apply here because $nlg^2(n) \,\epsilon\, o(n^{1 + \epsilon})$\\
\begin{table}[ht]
\begin{tabular}{ |c|c|c|c|c| }
 \hline
 level & instances & size/each & cost/each & total cost\\
 \cline{0-4}
 0 & 1 & n & $nlg^2n$ & $nlg^2n$\\
 \cline{0-4}
 1 & 2 & $\frac{n}{2}$ & $\frac{n}{2}lg^2\frac{n}{2}$ & $nlg^2\frac{n}{2}$ \\
 \cline{0-4}
 i & $2^i$ & $\frac{n}{2^i}$ & $\frac{n}{2^i}lg^2\frac{n}{2^i}$ & $nlg^2\frac{n}{2^i}$\\
 \cline{0-4}
 k & $2^k$ & $\frac{n}{2^k}$ & $c_0$ & $c_02^k$\\
 \hline
\end{tabular}
\end{table}\\
$c_0n + n\sum_{i=0}^{k-1} lg^2\frac{n}{2^i}\,=\,c_0n + n\sum_{i=0}^{k-1} lg^2n-2ilgn+i^2\,=\,c_0n + knlg^2n + n\sum_{i=0}^{k-1} i^2-2ilgn\,=\,c_0n + knlg^2n + \frac{n(k-1)(k)(2k-1)}{6} - nlgn(k)(k-1)$\\\\
With $k=lg(n)$ we have $T(n) = \Theta(nlg^3n)$\\\\
c) The Master Theorem does not apply here because $\frac{n^2}{lg(n)} \,\epsilon\, o(n^{2 + \epsilon})$\\
\begin{table}[ht]
\begin{tabular}{ |c|c|c|c|c| }
 \hline
 level & instances & size/each & cost/each & total cost\\
 \cline{0-4}
 0 & 1 & n & $\frac{n^2}{lg(n)}$ & $\frac{n^2}{lg(n)}$\\
 \cline{0-4}
 1 & 4 & $\frac{n}{2}$ & $\frac{n^2}{4lg\frac{n}{2}}$ & $\frac{n^2}{lg\frac{n}{2}}$ \\
 \cline{0-4}
 i & $4^i$ & $\frac{n}{2^i}$ & $\frac{n^2}{4^ilg\frac{n}{2^i}}$ & $\frac{n^2}{lg\frac{n}{2^i}}$\\
 \cline{0-4}
 k & $4^k$ & $\frac{n}{2^k}$ & $c_0$ & $c_04^k$\\
 \hline
\end{tabular}
\end{table}\\
$c_0n^2 + n^2\sum_{i=0}^{k-1}\frac{1}{lgn-i}\,=\,c_0n^2 - n^2\sum_{i=0}^{\floor*{lgn}}\frac{1}{i-lgn} - n^2\sum_{i=\ceil*{lgn}}^{k-1}\frac{1}{i-lgn}\,=\,c_0n^2 - c_1n^2 - c_2n^2$\\\\
The terms in the second term $n^2\sum_{i=0}^{\floor*{lgn}}$ will converge to 0 after a constant number of terms so it is bounded by the constant $c_1$.  For the third term, since $\frac{1}{i-lgn}$ is monotonically increasing starting from $i=\ceil*{lgn}$ we can take the integral $\int_{\ceil*{lgn}}^{k}\frac{1}{i-lgn}\,=\,ln(lgn-lgn) - ln(\ceil*{lgn}-lgn)$ which is approximately 0.  Therefore $T(n)=\Theta(n^2)$\\\\
d) Case 3 of the Master Theorem applies $\frac{n^\frac{5}{4}}{lg^2n}\,=\,\Omega(n^{log_45+\epsilon})$\\
$\lim_{n \to \infty}\frac{n^{1.16+\epsilon}}{\frac{n^{1.25}}{lg^2n}}\,=\,\frac{n^{1.16+\epsilon}lg^2n}{n^{1.25}}\,=\,\frac{lg^2n}{n^{.09-\epsilon}}$\\
$\frac{\frac{d}{dn}lg^2n}{\frac{d}{dn}n^{.09-\epsilon}}\,=\,\frac{2lgn}{n.09n^{-.91-\epsilon}ln2}\,=\,\frac{2lgn}{.09n^{.09-\epsilon}ln2}$\\
$\frac{\frac{d}{dn}2lgn}{\frac{d}{dn}.09n^{.09-\epsilon}ln2}\,=\,\frac{2}{.09^2n^{.09-\epsilon}2ln2}$\\
$\lim_{n \to \infty}\frac{2}{.09^2n^{.09-\epsilon}2ln2}=0$ for $\epsilon<.09$, therefore $\frac{n^\frac{5}{4}}{lg^2n}\,=\,\omega(n^{log_45+\epsilon})$ and $T(n)=\Theta(\frac{n^\frac{5}{4}}{lg^2n})$\\
\section*{Problem 4}
a) Correctness: 
\begin{itemize}
    \item Initialization: Cruel does not call Unusual until n=2, when both halves of A are trivially sorted. 
    
    \item Maintenance: Unusual then gradually sorts A as the stack of Cruel unwinds and n grows. The for loop swaps the second quarter of the array with the 
    third quarter of the array.  The recursive calls to Unusual maintain the invariant that each half is sorted, as the quarters are now 
    halves for the recursive calls.
    
    \item Termination: As the stack of Cruel unwinds, Unusual gradually sorts A in $n_i=2^i$ chunks, until i=k where $n=2^k$.  At this point the entire array
    has been sorted.
\end{itemize}
b) The array [3 4 1 2] would not sort correctly if the for loop were removed.  Unusual would return [3 1 4 2]\\\\
c) The array [3 4 1 2] would not sort correctly if the last 2 lines were swapped.  Unusual would return [1 3 2 4]\\\\
d) $T(n)=3T(\frac{n}{2})+\frac{n}{4}+1$ and $T(2)=3$\\
$f(n)=\frac{n}{4}+1$ and $n^{log_ba}=n^{lg3}=n^{1.58}$\\
Case 1 of the Master Theorem applies $\frac{n}{4}\,=\,O(n^{1.58-\epsilon})$\\
$\lim_{n \to \infty}\frac{4n^{1.58}}{nn^{\epsilon}}\,=\,4n^{.58-\epsilon}\,=\,\infty$ for $\epsilon<.58$\\
Therefore $T(n)=\Theta(n^{lg3})$\\\\
e)$T(n)=2T(\frac{n}{2})+\Theta(n^{lg3})$
\end{document}