\documentclass{article}
\usepackage{amsmath}
\usepackage[utf8]{inputenc}
\usepackage{titlesec}
\usepackage{fancyhdr}
 
\pagestyle{fancy}
\fancyhf{}
\rhead{\textit{CSC505 HW5}}
\lhead{\textit{Liam Adams}}
\cfoot{\thepage}
\renewcommand{\headrulewidth}{0pt}
%\newcommand{\sectionbreak}{\clearpage}

\begin{document}
\section*{Problem 1}
a) To prove \textit{M1-Sat} is NP-Complete we first must prove \textit{M1-Sat} $\epsilon NP$.  A boolean formula $\phi$ with $n$ boolean variables and $m$ boolean connectives can be encoded in a length that is polynomial in $n+m$. Now we need to show that a certificate consisting of a satisfying assigment for a formula $\phi$ can be verified in polynomial time.  This can be done by simply replacing each variable in the formula with its corresponding value and evaluating the expression, which can be done in polynomial time. If the expression evaluates to 1 and the number of true variables in the certificate is $\leq k$ then the algorithm has verified the formula and \text{M1-Sat} $\epsilon NP$.\\\\
Next we need to show that another NP Complete problem can be reduced to \textit{M1-Sat}. I will try to prove that the general formula satisfiability problem from pages 1079-1081 is reducible to the \textit{M1-Sat} problem $SAT \leq_p$ \textit{M1-Sat}.  First we construct a binary parse tree for the input formula $\phi$, with variables as leaves and connectives as internal nodes. The total number of clauses in our transformed formula needs to be less than $k$, so we pick a constant $c$ so that the number of variables $n$ divided by $c$ is less than $k$. Choosing $c$ will also ensure the truth table we construct later can be evaluated in polynomial time.\\\\ 
Then to make sure we have fewer than $k$ clauses we will parenthesize large clauses in $\phi$ so that there are only $c$ variables within the parenthesis, and therefore nodes in the tree will have at most $c$ children. Now we introduce a variable $y_i$ for the output of each internal node and rewrite $\phi$ as the AND of the root variable and a conjunction of clauses describing the operation of each node. We call this rewritten formula $\phi'$, and it is a conjunction of clauses $\phi'_i$ ANDed together. To make sure each clause only contains ORs we convert each clause to disjunctive normal form (an OR of ANDs) by creating a truth table for each clause and using the truth table entries that evaluate to 0. This will create a chain of disjunctive clauses of length equal to the number of 0 values in the truth table. Then use DeMorgan's Law to convert it to conjunctive normal form.\\\\
Now we have converted $\phi'$ into conjunctive normal form $\phi_{cnf}$. Now since we've performed legal boolean transformations of $\phi$ to get $\phi_{cnf}$, they are algebraically equivalent and $\phi_{cnf}$ is satisfiable if and only if $\phi$ is satisfiable. Now we must show that the reduction can be performed in polynomial time. We can introduce at most $2^c$ extra clauses in the transformation, and the truth table we need to evaluate has at most $2^c$ rows. Our choice of $c$ is less than $k$, so as long as $k$ can be encoded polynomially, $c$ also will be.  Thus the size of $\phi_{cnf}$ will be polynomial in the length $n+m$ of the original formula and the reduction algorithm will take time polynomial in the length of the original formula. Now we have shown \textit{M1-Sat} is NP Complete because there is a non deterministic algorithm that can verify a certificate for \textit{M1-Sat} in polynomial time and \textit{M1-Sat} is reducible from $SAT$, with both being algebraically equivalent to one another.\\\\
b) If you know $p$ is the total number of variables with a positive occurrence and $p \leq k$ you can set all the $p$ variables with a positive occurrence to $true$ and all the $n-p$ variables to $false$.  If the output of the formula is 1 \textit{M1-Sat} returns $yes$ because the number of true variables $p$ is less than $k$.  Otherwise return no.  If $p > k$, set all the $p$ variables with a positive occurrence to $false$ and the $n - p$ variables with a negative occurrence to $true$.  If the output of the formula is 1 than \textit{M1-Sat} returns $yes$ because the number of true variables, $n - p$, is less than $k$. Otherwise return no.  It takes polynomial time in $n$ to set each variables and evaluate the expression.\\\\
\section*{Problem 2}
a) Given a set of n tasks $a_1 ... a_n$, with a corresponding time, deadline, and profit, can you earn a total profit $P$ by deadline $D$?\\
b) There are $n$ tasks, a total deadline of $\sum_{i=0}^nt_i = D_{max}$, and a total possible profit of $\sum_{i=0}^np_i = P_{max}$ summed over all the $n$ tasks, so the problem can be encoded in a length that is polynomial in $D$. To show this is NP Complete first we must show it is a member of NP. We must show that a certificate consisting of a sequence of tasks earns profit $P$ by deadline $D$ can be verified in polynomial time.  This can be verified in polynomial time by adding up the profits and times of the tasks and comparing them to $P$ and $D$.\\
Now we must show that an NP Complete problem can be reduced to the scheduling problem. I will try to show that the M1-Sat problem can be reduced to the Scheduling problem. Given a CNF formula $\phi$ over $n$ variables $x_1 ... x_n$ and $k$ clauses $C_1 ... C_k$ the reduction algorithm constructs an instance of the scheduling problem $(S, P, D)$ such that $\phi$ is satisfiable with less than $k$ true variables if and only if there exists a subset of $S$ whose profit is at least $P$ and sum of times less than $D$. We will construct a table with $n+k$ columns and $2n + 2k$ rows. The rows correspond to numbers in $S$, so there are 2 numbers in $S$ for each variable $x_i$ ($x_i$ and $x_i'$) and 2 numbers for each clause $c_i$. The numbers represented by the rows are in base $2n+1$, each digit corresponds to a column which is either a variable or a clause. The columns sum to an $n+k$ digit number corresponding to $P$. The rows sum to a $2n + 2k$ digit number corresponding to $D$. Regarding the columns, the least significant $k$ digits are the clauses, and the most significant $n$ digits are the variables.  Now use the set of variable value pairs the satisfy the formula to construct the numbers in each row. You will only use $n$ rows because you cannot have both $x_i$ and $x_i'$ in your solution to the formula. You put a 1 in one of the most significant $n$ columns matching the row selected if the variable is positive, and a 2 if the variable is negative. Also put a 1 or 2 in the least significant $k$ columns in which that variable value appears. The most significant $n$ columns will sum to 2 and the least significant $k$ columns will sum to $2n$.  You use the bottom $2k$ rows to make the sum equal $2n$ if the top $2n$ do not make the sum that high. The number at the bottom corresponds to $P$ and the number to the right of the rows corresponds to $D$.\\\\ 
This reduction can be done in polynomial time because the set $S$ contains $2n+2k$ values, each having $n+k$ digits, and the time to produce each is polynomial in $n+k$.  $P$ has $n+k$ digits, and $D$ has $2n+2k$ digits, and the reduction produces each in constant time. Now we have to show that $\phi$ is satisfiable if and only if there exists a subset $S'$ of $S$ whose profit is at least $P$ and deadline at most $D$. Include in $S'$ only those variables $x_i$ or $x_i'$ that are 1.  At most $k$ for which $x_i=1$. The sum of these most significant $n$ columns will be 2, with the bottom $2k$ rows adding a 1 if necessary. Because each clause is satisfied, the clause contains some variable with the value of 1. Therefore each of the $k$ least significant digits has at least 1 unit contributed by a variable, and up to $2n$ units. So each clause labeled digit has a sum between 1 and $2n$. We use the bottom $2k$ to ensure that each of the $k$ least significant digits sum to $2n$. Since we have matched the target in all digits of the sum, the profits of $S'$ sum to $P$.  An analogous proof can be made for the sum of the rows and $D$.\\
Now suppose there is a subset $S'$ that sums to $P$ and less than $D$. The subset must include exactly one of $x_i$ and $x_i'$ for each $i = 1...n$, otherwise the digits labeled by the variables would not sum to 2. If $x_i$ is in $S$ we set $x_i=1$, if $x_i'$ is in $S$ we set $x_i'=2$. Every clause $C_j$ for $j=1..k$ is satisfied by this assignment. To achieve a sum of $2n$ in the digits labeled by $C_j$, the subset $S'$ must include at least 1 $x_i$ or $x_i'$ value that has a 1 or 2 in the digit labeled by $C_j$, since the bottom $2k$ variables cannot sum to $2n$. If $S'$ includes a $x_i$ with a 1 in $C_j$'s position, then that literal appears in the clause $C_j$, satisfying the clause. The same applies when $x_i'=1$. Thus all the clauses are satisfied, completing the proof.
\end{document}